\chapter{Wyniki eksperymentów}
\label{c5}

\section{Wstęp}
\label{c51}
W poniższym rozdziale opisano przebieg przeprowadzonych eksperymentów oraz przedstawiono uzyskane wyniki wraz z ich interpretacją. Algorytmy testowane były na tych samych zbiorach danych. Tworzenie tych zbiorów odbywało się z wykorzystaniem generatora przyjmującego jako parametry liczbę transakcji (np. 10 tys., 100 tys.), średnią liczbę elementów w transakcji (np. 6), liczbę wzorców do odkrycia (np. 500), średni rozmiar wzorców częstych do odkrycia (np. 3), liczbę różnych elementów występujących w transakcjach (np. 1000, 10000) oraz nazwę pliku wyjściowego. Wygenerowany plik jest importowany do bazy danych, do której odwołuje się aplikacja podczas wykonywania algorytmów. Dane w bazie składowane są w jednej tabeli w postaci par \((id transakcji, id elementu)\). Dlatego też po odczycie tej tabeli składane są transakcje wykorzystywane w dalszym przetwarzaniu. 

\section{Opis infrastruktury}
\label{c52}
Algorytmy napisane zostały w języku Java, z wykorzystaniem narzędzia Maven oraz środowiska programistycznego Eclipse. Dane testowe generowane były za pomocą generatora GEN (\cite{AgrawalGEN}), a następnie wczytywane do bazy PostgreSQL, z której korzystała aplikacja. Testy przeprowadzone zostały na komputerze HP Envy 14 Notebook PC, z procesorem Intel Core i5-2410M 2x2.30GHz oraz 8GB pamięci RAM, pracującym pod kontrola systemu operacyjnego Microsoft Windows 7. 

\section{S2}
\label{c53}
