\chapter{Podstawowe pojęcia i definicje}
\label{c2}

\section{Wstęp}
\label{c21}

W poniższym rozdziale omówiono podstawowe pojęcia i definicje wykorzystywane w pracy. 

\section{Lista pojęć i definicji}
\label{c22}

\subsection{Transakcja i element transakcji}
\label{c221}
Danymi wejściowymi dla odkrywania zbiorów częstych i reguł asocjacyjnych jest zbiór transakcji zdefiniowanych na zbiorze elementów. Tymi elementami mogą być produkty w sklepie, usługi, książki etc. Ważne jest, aby te elementy można było w łatwy sposób od siebie odróżnić. Jeśli \( I = \{ i_1, i_2, \cdots, i_n \}\)  (ang. \textit {item base} ), to zbiór wszystkich możliwych elementów, to dowolny niepusty podzbiór X zbioru \(X\subseteq I\) nazywamy transakcją (ang. \textit{itemset}). Natomiast zbiór elementów o mocy \textit{k}, to taki zbiór, który posiada dokładnie k elementów (ang. \textit{k-itemset}). Mówi się, że 

Transakcja jest zatem przykładowym zbiorem elementów, np. zbiorem produktów, które zostały kupione przez danego klienta. Jako że transakcje mogą się powtarzać (może istnieć kilku klientów, którzy kupili dokładnie takie same produkty), to nie ma możliwości żeby zamodelować wszystkie możliwe transakcje (koszyki). Wynika to z tego, że elementy w zbiorze nie mogą się powtarzać. Problem ten znalazł kilka rozwiązań. Należy do nich zamodelowanie wszystkich transakcji jako multizbioru (uogólnienie pojęcia zbioru, w którym w odróżnieniu od klasycznych zbiorów jeden element może występować wiele razy) albo jako wektora (elementy na różnych pozycjach mogą być takie same, ale wyróżnia je położenie). Innym - choć podobnym do wspomnianego wyżej zastosowania wektora - rozwiązaniem jest rozszerzenie każdej transakcji o unikalny identyfikator. Kolejną możliwością jest wykorzystanie zbioru unikalnych transakcji, z tą różnicą, że do każdej transakcji przypisany jest licznik mający za zadanie zliczanie wystąpień.

Należy także zwrócić uwagę, że w większości rozważanych przypadków nie są znane wszystkie elementy, jakie mogą znaleźć się w zbiorze \(I\). Przyjmuje się wówczas, że ten zbiór jest sumą elementów występujacych we wszystkich transakcjach.

\subsection{Reguła asocjacyjna}
\label{c222}
Reguła asocjacyjna jest implikacją, która daje możliwość przewidywania jednoczesnego wystąpienia dwóch zjawisk i zachowań, współzależnych od siebie. Innymi słowy jest to schemat, pozwalający - z określonym prawdopodobieństwem - założyć, że jeśli nastąpiło zdarzenie A, to nastąpi również zdarzenie B. W kontekście problemu koszyka zakupów sprowadza się do reguł w stylu: \textit{Jeżeli klient kupił pieluszki, to (z określonym prawdopodobieństem) kupi też piwo.}

\subsection{Wsparcie transakcji}
\label{c223}
Jeśli \(T\) oznacza jedną z transakcji w zbiorze wszystkich transakcji \(D\), to (bezwzgędne) wsparcie tej transakcji jest równe \(U\) - liczbie wystąpień \(T\) w zbiorze \(D\). Wsparcie względne jest to z kolei procent (lub ułamek) transakcji w zbiorze \(D\), które zawierają \(T\). Obliczamy ze wzoru \[sup_{rel}(T)=\nolinebreak\frac{|U|}{|D|}*100\%\]. Dla algorytmu Apriori określa się próg minimalnego wsparcia \(minsup\), który również może być wyrażony w dwojakiej postaci - jako liczba wystąpień lub procent wszystkich transakcji. W poszukiwaniu zbiorów częstych interesujące są tylko te reguły, dla których \(sup(T) \geq minsup \), gdzie \(sup(T)\), to przyjęty w pracy sposób zapisu wsparcia transakcji \(T\) w rozważanym zbiorze transakcji \(D\).

\subsection{Ufność reguły asocjacyjnej}
\label{c224}
Ufność reguły asocjacyjnej jest miarą jakości danej reguły. Miara ta została przedstawiona przez autorów algorytmu Apriori \cite{Agrawal1994}. Dla reguły asocjacyjnej postaci \(R = "X\rightarrow Y"\) (gdzie \(X\) i \(Y\) to zbiory elementów) ufność wyraża się jako stosunek wspracia sumy wsystkich elementów występujących w regule (w tym przypadku \(sup(X \cup Y)\)) do wstparcia poprzednika reguły (tutaj \(sup(X)\)). 
\[conf(R) = \frac{sup(X \cup Y)}{sup(X)}\]
Należy dodać, że nie ma znaczenia czy wykorzystywane jest wsparcie aboslutne czy relatywne. Istotne jest natomiast to, aby w zarówno dla licznika i mianowika wykorzystany był ten sam typ wsparcia.
Z powyższego wzoru wynika, że ufność reguły asocjacyjnej, to stosunek liczby przypadków, w których jest ona poprawna, do wszystkich przypdków gdzie mogłaby zostać zastosowana.
Przykład: \(R = wino \wedge chleb \rightarrow ser\) - jeśli klient kupuje wino i chleb, to ta reguła ma zastowanie i mówi, że można oczekiwać, że dany klient kupi również ser. Jest możliwe, że ta reguła - dla danego klienta - będzie poprawna lub nie. Interesują informacją jest to jak dobra jest reguła, czyli jak często jest poprwana (jak często klient, króty kupuje wino i chleb kupuje również ser). Taką właśnie informację uzyskuje się poprzez obliczenie ufności reguły asocacyjnej. Oczywiście w przypadku gdy klient nie kupił chleba lub/i wina, to reguła nie znajduje zastowania, a dana transakcja nie wpływa na \(conf(R)\). 

\subsection{Wsparcie reguły asocjacyjnej}
\label{c225}
Wsparcie reguły asocjacyjnej postaci \(A \cup B \rightarrow C\) odpowiada wsparciu zbioru \(S = \{A, B, C\}\) (\cite{Agrawal}).Miara ta informuje o tym jak często dana reguła jest prawidłowa. Nieco odmienna definicja została przedstawiona i wykorzystana w \cite{Borgelt}. Różnica polega na tym, że wpsarcie wyrażone jest jako liczba przypdków, w których reguła jest stosowalna. Zatem dla powyżej postaci byłoby to \(S = \{A, B\}\), nawet jeśli reguła może okazać się fałszywa. 
Wparcie może być stosowane do filtrowania. Dla ustalonego \(minsup\) szuka się tylko takich reguł, których wsparcie jest nie mniejsze od \(minsup\). Oznacza to, że interesujące są tylko te reguły, które wystąpiły co najmniej daną liczbę razy.
W algorytmach wyszukiwania reguł asocjacyjnych stosuje się progi mnimalnego wsparcia oraz mnimalnej ufności. Dzięki temu w otrzymanych wynikach nie są uwzględnione mało wartościowe reguły.

\subsection{Zbiór częsty}
\label{c226}
Zbiorem częstym nazywamy taki niepusty podzbiór zbioru \(I\), dla którego wsparcie jest równe co najmniej wartości \(minsup\).

\subsection{Zbiór domknięty}
\label{c227}
Zbiorem domkniętym nazywamy taki zbiór częsty, dla którego nie istnieje żaden nadzbiór mający dokładnie takie samo wsparcie.

\subsection{Zbiór maksymalny}
\label{c228}
Zbiorem maksymalnym nazywamy taki zbiór częsty, dla którego nie istnieje żaden nadzbiór, który byłby zbiorem częstym.

\subsection{Zapytanie eksploracyjne}
\label{c229}

\subsection{Zbiór elementarnych predykatów selekcji danych dla zbioru zapytań eksploracyjnych}
\label{c2210}