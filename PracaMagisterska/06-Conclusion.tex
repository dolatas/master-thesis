\chapter{Wnioski i uwagi}

Efektem pracy są implementacje dwóch algorytmów wykonania zbioru zapytań odkrywających zbiory częste - które dotychczas implementowane były na drzewie haszowym - z wykorzystaniem drzew prefiksowych. Opisana została problematyka zagadnienia oraz dokonano przeglądu istniejących rozwiązań związanych z tematem pracy. Przeprowadzone zostały również testy tych algorytmów oraz przedstawiona została analiza uzyskanych wyników. 

Przeprowadzone eksperymenty pokazują, że możliwa jest adaptacja algorytmów przetwarzania zbiorów zapytań eksploracyjnych (Common Counting \cite{WojciechowskiCC} i Common Candidate Tree \cite{WojciechowskiCCT}) opartych na oryginalnym Apriori (\cite{Agrawal}) do rozwiązań korzystających z szybszych wersji algorytmu Apriori wykorzystujących drzewa prefiksowe. Dla CCT nie było to zadanie tak trywialne jak w wypadku CC, ale jednak po rozwiązaniu problemów implementacyjnych osiągnięto efekt podobny jak raportowany wcześniej dla drzew haszowych. Utrudnieniem w przypadku CCTP okazał się sposób analizy i przeglądania drzewa prefiksowego przez użyty algorytm Apriori. Z testów wynika, że stosowanie CCP jest wskazane dla przetwarzania zbioru zapytań eksploracyjnych niezależnie stopnia nakładania się danych czy też innych parametrów przetwarzania. W najgorszym przypadku uzyskany czas będzie porównywalny z czasem algorytmu sekwencyjnego. Należy jedynie pamiętać o ograniczeniach pamięciowych, które wymagają od CC przechowywania wielu drzew w pamięci, podczas gdy dla (SEQ) analizowane jest tyko jedno drzewo w danym momencie. Natomiast wykorzystanie CCTP przynosi najwięcej korzyści w momencie gdy partycje danych, do których odwołują się zapytania mają wysoki współczynnik nakładania się, w przeciwnym wypadku istnieje ryzyko wydłużenia czasu wykonania.

Adaptacje algorytmu Apriori oraz algorytmy przetwarzające zbiory zapytań eksploracyjnych nadal mogą być ulepszane i optymalizowane. Jednym z możliwych rozwiązań jest próba wykorzystania Common Counting w taki sposób, że na podstawie zbioru elementarnych predykatów selekcji danych dla zbioru zapytań eksploracyjnych \(S\) generowane są rozłączne zapytania \(dmq_i\), dla których algorytm wykonywany jest w ten sam sposób jaki opisano w \ref{c43}. Ostatnim krokiem byłaby wówczas integracja znalezionych zbiorów częstych w celu uzyskania odpowiedzi na oryginalne zapytania ze zbioru \(DMQ\). Takie podejście zmniejszyłoby rozmiary drzew przechowujących kandydatów podczas przetwarzania. Należałoby sprawdzić czy zysk ten jest większy niż koszt operacji generowania zbioru rozłącznych zapytań \(DMQ\). 

Innym zagadnieniem jest problem optymalizacji generowania kandydatów dla CCTP. Sortowanie wartości podczas dodawania jest czasochłonne, mimo że porównywane są jedynie dwie wartości zbioru elementów, a nie cała lista wartości. Mimo prób zastosowania innych struktur (m.in. zbioru czy też drzew haszowych) podczas tej operacji nie udało się uzyskać lepszego wyniku niż dla ostatecznie zastosowanego rozwiązania. 

Pewne jest, że badany obszar pozostawia bardzo dużo możliwości i perspektyw do prowadzenia dalszych prac nad optymalizacją wykonania zarówno zbioru zapytań eksploracyjnych, jak i samego algorytmu Apriori. 