% Szkielet dla pracy pisanej w języku angielskim.

\documentclass[polish,a4paper,twoside]{ppfcmthesis}

\usepackage[utf8]{inputenc}
\usepackage[OT4]{fontenc}
\usepackage{listings}
\usepackage{amsfonts}
\usepackage{amsmath}

\author{Szymon Dolata}                              % Your name comes here
\title{Integracja drzew prefiksowych w przetwarzaniu zbiorów zapytań eksploracyjnych algorytmem Apriori}        % Note how we protect the final title phrase from breaking
\ppsupervisor{Dr inż. Marek Wojciechowski} % Your supervisor comes here.
\ppyear{2014}                                         % Year of final submission (not graduation!)

\begin{document}

% Front matter starts here
\frontmatter\pagestyle{empty}%
\maketitle\cleardoublepage%

\pagebreak

% Blank info page for "karta dyplomowa"
\thispagestyle{empty}\vspace*{\fill}%
\begin{center}Tutaj przychodzi karta pracy dyplomowej;\\oryginał wstawiamy do wersji dla archiwum PP, w pozostałych kopiach wstawiamy ksero.\end{center}%
\vfill\cleardoublepage%

\pagebreak

% Table of contents.
\pagenumbering{Roman}\pagestyle{ppfcmthesis}%
\tableofcontents* \cleardoublepage%

% Main content of your thesis starts here.
\mainmatter%
\chapter{Wstęp}
\label{c1}

\section{Integracja drzew prefiksowych w przetwarzaniu zbiorów zapytań eksploracyjnych algorytmem Apriori}
\label{c11}


Odkrywanie zbiorów częstych i generowanie na ich podstawie reguł asocjacyjnych, to problem sformułowany w kontekście analizy koszyka zakupów. Głównym celem jest szukanie prawidłowości w zachowaniu klientów supermarketów. Szybko znalazł on również zastosowanie w wielu innych dziedzinach, takich jak chociażby analiza działalności firm wysyłkowych, sklepów internetowych etc. Z wykorzystaniem znalezionych zbiorów częstych i wygenerowanych reguł dąży się do tego aby możn było wynioskować (z dużym prawdopodobieństwem), że niektóre produkty współwystępują ze sobą. Informacje takie, zwłaszcza jeśli wyrażone w formie zasad, często mogą być stosowane w celu zwiększenia sprzedanych danych produktów - na przykład poprzez odpowiednie rozmieszczenie ich na półkach w supermarkecie lub na stronach katalogu wysyłkowego (umieszczenie obok siebie może zachęcić jeszcze więcej klientów do zakupu ich razem) lub poprzez bezpośrednie sugerowanie klientom produktów, którymi mogą być zainteresowani. 

Oczywistym jest, że należy szukać tylko takich reguł asocjacyjnych, które są wiarygodne i niosą ze sobą jakąś informację. Istnieją wskaźniki służące do oceny tychże reguł. Zostały one omówione bardziej szczegółowo w rozdziale 2. 

Głównym problemem indukcji reguł asocjacyjnych jest to, że istnieje bardzo wiele możliwości. Przykładowo w zakresie produktów z supermarketu, których może być nawet kilka tysięcy, istnieją miliardy możliwych reguł. Tak ogromna ilość nie może być przetwarzana sekwencyjnie. Dlatego potrzebne są wydajne algorytmy, które ograniczają przestrzeń wyszukiwania i sprawdzają jedynie podzbiór wszystkich reguł. Jednym z takich algorytmów jest Apriori opracowany przez \cite{Agrawal}.

Podstawowy algorytm Apriori trzyma kandydatów w drzewie haszowym. W ostatnich latach zaproponowane zostały metody Common Counting (\cite{WojciechowskiCC}) oraz Common Candidate Tree (\cite{WojciechowskiCCT}). Są one wynikiem badań nad optymalizacją wykonania kilku zadań Apriori uruchomionych współbieżnie na nakładających się podzbiorach tabeli z danymi. Metody sprowadzały się do:\newline
- Integracji odczytów współdzielonych danych z dysku;\newline
- Integracji drzew haszowych w jedno drzewo gdzie kandydaci mają kilka liczników (po jednym dla zadania eksploracji).\newline
W praktyce jednak lepsze okazały się implementacje Apriori gdzie drzewo haszowe zastąpiono znacznie prostszą strukturą drzewa prefiksowego. Powstało kilka rozwiązań wykorzystujacych tę strukturę: Borgelt, Bodon, Goethals. Jednak do tej pory nie została zaimplementowana modyfikacja Common Counting i Common Candidate Tree dla Apriori z drzewem prefiksowym i to właśnie jest celem tej pracy. 

\section{Cel i zakres pracy}
\label{c12}

Tak jak wspomniano, ogolnym celem pracy jest implementacja dwóch algorytmów wykonania zbioru zapytań odkrywających zbiory częste - które dotychczas implementowne były na drzewie haszowym - z wykorzystaniem drzew prefiksowych. \newline
Na ten ogólny cel pracy składają się następujące cele szczegółowe:
- przedstawienie, analiza i porównanie istniejących rozwiązań dotyczących tematyki pracy
- implementacja modyfikacji Common Counting i Common Candidate Tree dla Apriori z drzewem prefiksowym;\newline
- przetesotowanie wydajności zaimplementowanych algorytmów.\newline

\section{Opis infrastruktury}
\label{c13}
Algorytmy napisane zostały w języku Java, z wykorzystaniem narzędzia Maven oraz środowiska programistycznego Eclipse. Dane testowe generowane były za pomocą generatora GEN (\cite{AgrawalGEN}), a następnie wczytywane do bazy PostgreSQL, z której korzystała aplikacja. Tetsy przeprowadzone zostały na komputerze HP Envy 14 Notebook PC, z procesorem Intel Core i5-2410M 2x2.30GHz oraz 8GB pamięci RAM, pracujacym pod kontrola systemu operacyjnego Microsoft Windows 7. 
	
\section{Struktura pracy}
\label{c14}   
Struktura pracy jest następująca:\newline
-- w rozdziale 2 omówiono podstawowe pojęcia i definicje wykorzystywane w pracy;\newline
-- w rozdziale 3 przedstawiono istniejące rozwiązania i algorytmy, związane z tematem pracy;\newline
-- w rozdziale 4 przedstawiono ideę, opis i cechy alogrytmów;\newline
-- w rozdziale 5 przeanalizowano działanie algorytmów dla różnych parametrów i danych wejściowych;\newline
-- w rozdziale 6 przedstawiono wnioski i uwagi do pracy.
\chapter{Podstawowe pojęcia i definicje}
\label{c2}

\section{Wstęp}
\label{c21}

W poniższym rozdziale przedstawiono podstawowe pojęcia i definicje wykorzystywane w pracy. Związane są one z omawianym tematem. Na początku opisano terminy niezbędne dla sformułowania problemu. Ostatnie dwa pojęcia są natomiast kluczowe dla rozważanego w pracy problemu badawczego.

\section{Lista pojęć i definicji}
\label{c22}

\subsection{Transakcja i element transakcji}
\label{c221}
Danymi wejściowymi dla odkrywania zbiorów częstych i reguł asocjacyjnych jest zbiór transakcji zdefiniowanych na zbiorze elementów. Tymi elementami mogą być produkty w sklepie, usługi, książki etc. Ważne jest, aby te elementy można było w łatwy sposób od siebie odróżnić. Jeśli \( I = \{ i_1, i_2, \cdots, i_n \}\)  (ang. \textit {item base} ), to zbiór wszystkich możliwych elementów, to dowolny niepusty podzbiór X zbioru \(X\subseteq I\) nazywamy zbiorem elementów (ang. \textit{itemset}). Natomiast zbiór elementów o mocy \textit{k}, to taki zbiór, który posiada dokładnie k elementów (ang. \textit{k-itemset}). 

Transakcja jest natomiast przykładowym zbiorem elementów, np. zbiorem produktów, które zostały kupione przez danego klienta. Jako że transakcje mogą się powtarzać (może istnieć kilku klientów, którzy kupili dokładnie takie same produkty) każda transakcja w bazie danych ma przypisany unikalny identyfikator \(TID\). Dzięki temu nie ma problemu występowania duplikatów, pomimo tego, że mogą występować transakcje zawierające ten sam zestaw elementów. 

Należy także zwrócić uwagę, że w większości rozważanych przypadków nie są znane wszystkie elementy, jakie mogą znaleźć się w zbiorze \(I\). Przyjmuje się wówczas, że ten zbiór jest sumą elementów występujących we wszystkich transakcjach.

\subsection{Reguła asocjacyjna}
\label{c222}
Reguła asocjacyjna jest implikacją, która daje możliwość przewidywania jednoczesnego wystąpienia dwóch zjawisk i zachowań, współzależnych od siebie. Innymi słowy jest to schemat, pozwalający - z określonym prawdopodobieństwem - założyć, że jeśli nastąpiło zdarzenie A, to nastąpi również zdarzenie B. Reguła może zostać wyrażona w postaci \(R = "X\rightarrow Y"\) (gdzie \(X\) i \(Y\) to zbiory elementów). W kontekście problemu koszyka zakupów sprowadza się do reguł w stylu: \textit{Jeżeli klient kupił pieluszki, to (z określonym prawdopodobieństwem) kupi też piwo.} 

\subsection{Wsparcie zbioru elementów}
\label{c223}
Jeśli \(X\) oznacza dowolny podzbiór \(I\), to (bezwzględne) wsparcie tego zbioru jest równe \(U\) - liczbie transakcji \(T\) w bazie danych transakcji \(D\), które zawierają wszystkie elementy danego zbioru \(X\). Wsparcie względne jest to z kolei procent (lub ułamek) transakcji w \(D\), które zawierają \(X\). Obliczamy ze wzoru \[sup_{rel}(X)=\nolinebreak\frac{|U|}{|D|}*100\%\] Dla algorytmu Apriori określa się próg minimalnego wsparcia \(minsup\), który również może być wyrażony w dwojakiej postaci - jako liczba wystąpień lub procent wszystkich transakcji. W poszukiwaniu zbiorów częstych interesujące są tylko te reguły, dla których \(sup(X) \geq minsup \), gdzie \(sup(X)\), to przyjęty w pracy sposób zapisu wsparcia zbioru elementów.

\subsection{Ufność reguły asocjacyjnej}
\label{c224}
Ufność reguły asocjacyjnej jest miarą jakości danej reguły. Miara ta została przedstawiona przez autorów algorytmu Apriori \cite{Agrawal}. Dla reguły asocjacyjnej postaci \(R = "X\rightarrow Y"\) ufność wyraża się jako stosunek wsparcia sumy wszystkich elementów występujących w regule (w tym przypadku \(sup(X \cup Y)\)) do wsparcia poprzednika reguły (tutaj \(sup(X)\)). 
\[conf(R) = \frac{sup(X \cup Y)}{sup(X)}\]
Należy dodać, że nie ma znaczenia czy wykorzystywane jest wsparcie absolutne czy relatywne. Istotne jest natomiast to, aby w zarówno dla licznika i mianownika wykorzystany był ten sam typ wsparcia.
Z powyższego wzoru wynika, że ufność reguły asocjacyjnej, to stosunek liczby przypadków, w których jest ona poprawna, do wszystkich przypadków gdzie mogłaby zostać zastosowana.
Przykład: \(R = wino \wedge chleb \rightarrow ser\) - jeśli klient kupuje wino i chleb, to ta reguła ma zastosowanie i mówi, że można oczekiwać, że dany klient kupi również ser. Jest możliwe, że ta reguła - dla danego klienta - będzie poprawna lub nie. Interesują informacją jest to jak dobra jest reguła, czyli jak często jest poprawna (jak często klient, który kupuje wino i chleb kupuje również ser). Taką właśnie informację uzyskuje się poprzez obliczenie ufności reguły asocjacyjnej. Oczywiście w przypadku gdy klient nie kupił chleba lub/i wina, to reguła nie znajduje zastosowania, a dana transakcja nie wpływa na \(conf(R)\). 

\subsection{Wsparcie reguły asocjacyjnej}
\label{c225}
Wsparcie reguły asocjacyjnej postaci \(R = "X\rightarrow Y"\) odpowiada wsparciu sumy zbiorów \(X\) i \(Y\) (\cite{Agrawal}). Miara ta informuje o tym jak często dana reguła jest prawidłowa. Nieco odmienna definicja została przedstawiona i wykorzystana w \cite{Borgelt}. Różnica polega na tym, że wartość ta wyrażona jest jako liczba przypadków, w których reguła jest stosowalna (nawet jeśli reguła może okazać się fałszywa). Zatem dla powyżej postaci byłoby to wsparcie zbioru \(X\). 

Wparcie może być stosowane do filtrowania. Dla ustalonego \(minsup\) szuka się tylko takich reguł, których wsparcie jest nie mniejsze od \(minsup\). Oznacza to, że interesujące są tylko te reguły, które wystąpiły co najmniej daną liczbę razy.
W algorytmach wyszukiwania reguł asocjacyjnych stosuje się progi minimalnego wsparcia oraz minimalnej ufności. Dzięki temu w otrzymanych wynikach nie są uwzględnione mało wartościowe reguły.

\subsection{Zbiór częsty}
\label{c226}
Zbiorem częstym nazywamy taki niepusty podzbiór zbioru \(I\), dla którego wsparcie jest równe co najmniej wartości \(minsup\).

\subsection{Zbiór domknięty}
\label{c227}
Zbiorem domkniętym nazywamy taki zbiór częsty, dla którego nie istnieje żaden nadzbiór właściwy mający dokładnie takie samo wsparcie.

\subsection{Zbiór maksymalny}
\label{c228}
Zbiorem maksymalnym nazywamy taki zbiór częsty, dla którego nie istnieje żaden nadzbiór właściwy, który byłby zbiorem częstym.

\subsection{Zapytanie eksploracyjne}
\label{c229}
Zapytanie eksploracyjne jest uporządkowaną piątką \(dmq = (R, a, \Sigma, \Phi, minsup) \), gdzie \(R\) - relacja bazy danych, \(a\) - atrybut relacji \(R\), \(\Sigma\) - wyrażenie warunkowe dotyczące atrybutów \(R\) nazywane predykatem selekcji danych, \(\Phi\) - wyrażenie warunkowe dotyczące odkrywanych zbiorów częstych nazywane predykatem selekcji wzorców, \(minsup\) - próg minimalnego wsparcia. Wynikiem zapytania eksploracyjnego są zbiory częste odkryte w \(\pi _a \sigma _\Sigma R \), które spełniają predykat \(\Phi\) i posiadają \(wsparcie \geq minsup\) (\(\pi\) - relacyjna operacja projekcji, \(\sigma\) - relacyjna operacja selekcji).

\subsection{Zbiór elementarnych predykatów selekcji danych}
\label{c2210}
Zbiorem elementarnych predykatów selekcji danych dla zbioru zapytań eksploracyjnych \(DMQ = \{dmq_1, dmq_2, \dots, dmq_n\}\) nazywamy najmniej liczny zbiór \(S = \{s_1, s_2, \dots, s_k\}\) (\(s_i\) - \(i\)-ty predykat selekcji danych z relacji \(R\)), dla którego dla każdej pary \(u, v (u \neq v)\) zachodzi \(\sigma{s_u}R \cap \sigma{s_v}R = \emptyset\)  i dla każdego \(dmq_i\) istnieją liczby całkowite \(a, b, \dots, m\), takie że \(\sigma_{\Sigma_i}R = \sigma_{s_a}R \cup \sigma_{s_b}R \cup \dots \cup \sigma_{s_m}R\). Zbiór ten jest reprezentacją podziału bazy danych na partycje, które zostały wyznaczone przez nakładające się źródłowe zbiory danych zapytań.
\chapter{Podłoże teoretyczne}
\label{c3}

\section{Wstęp}
\label{c31}
Kolejny rozdział przedstawia aktualne metody i istniejące algorytmy związane z tematem pracy. Poza podstawowym algorytmem Apriori (\cite{Agrawal}), który używa drzew haszowych do przechwywania kandydatów, opisano trzy modyfikacje tego alogrytmu. Główna różnica polega na tym, że wykorzystują one inną strukturę, a mianowicie drzewa prefiksowe. Są to rozwiązania zaproponowane przez Christina Borgelta (\cite{Borgelt}), Ferenca Bodona (\cite{Bodon}) oraz Barta Goethalsa (\cite{Goethals}). Ze względu na wykorzystanie prostszej struktury okazały się one szybsze od standardowego algorytmu.
 
Innym problemem jest optymalizacja wykonania kilku zadań Apriori uruchomionych współbieżnie na nakładających się podzbiorach tabeli z danymi. Metody z tym związane to Common Counting (\cite{WojciechowskiCC}) i Common Candidate Tree (\cite{WojciechowskiCCT}). Oparte są one o implementację Apriori z zastosowaniem drzew haszowych. Brakuje jednak adaptacji tych algorytmów, polegającej na zmianie struktury na drzewa prefiksowe. Właśnie taka modyfikacja została wprowadzona, a uzyskane efekty opisano w kolejnych rozdziałach niniejszej pracy. 


\section{Przegląd istniejących rozwiązań}
\label{c32}

\subsection{Algorytm Apriori \cite{Agrawal}}
\label{c321}
Algorytm Apriori jest algorytmem eksploracji poziomej. Szuka zbiorów częstych o rozmiarach \(1, 2,\dots , k\). Algorytm rozpoczyna od zbiorów o rozmiarze 1 i następnie zwiększa ten rozmiar w kolejnych iteracjach. Elementy każdej transakcji są uporządkowane leksykograficznie - jeżeli nawet transakcje nie są posortowane, to krokiem wstępnym algorytmu może być leksykograficzne uporządkowanie elementów transakcji (\cite{Morzy}). Po pierwszym kroku zebrane są zatem wszystkie elementy występujące w transakcjach (w postaci zbiorów jednolementowych). Następnie sprawdzane jest, które z nich posiadają wsparcie nie mniejsze niż \(minsup\). Elementy niespełniające tego wymgania są odrzucane. Pozostałe służą do utworzenia dwuelementowych zbiorów kandydujących (ang. \textit{candidate itemsets}). Dla wygenerowanych zbiorów spradzane jest czy posiadają wsparcie równe co najmniej \(minsup\). Jeśli tak, to taki zbiór jest dodawany do listy zbiorów częstych i w kolejnej iteracji jest wykorzystywany (wraz z innymi zbiorami z tejże listy) do generowania zbiorów kandydatów o rozmiarze o 1 większym. Wsparcie zbiorów sprawdzane jest na podstawie odczytu danych z bazy danych. Algorytm zatrzymuje się gdy nie ma już możliwości generowania kolejnych zbiorów. W wyniku jego działania zwracana jest suma \(k\)-elementowych zbiorów częstych \((k = 1, 2,\dots)\), która może zostać wykozystana do generowania reguł asocjacyjnych.

\subsection{Algorytm Apriori - implementacja Christina Borgelta \cite{Borgelt}}
\label{c322}

\subsection{Algorytm Apriori - implementacja Ferenca Bodona \cite{Bodon}}
\label{c323}

\subsection{Algorytm Apriori - implementacja Barta Goethalsa \cite{Goethals}}
\label{c324}


\subsection{Common Counting \cite{WojciechowskiCC}}
\label{c325}
W metodzie Common Counting chodzi o równoległe wykonanie zbioru zapytań eksploracyjnych algorytmem Apriori z integracją porywających się fragmentów bazy danych. Na wejściu algorytm otrzymuje zbiór elementarnych predykatów selekcji danych dla zbioru zapytań eksploracyjnych \(DMQ\). Początkowo algorytm ustala zbiór wszystkich elementów, czyli takich, które wystąpiły w co najmniej jednej transakcji. W kolejnych krokach generowane są zbiory częste odzielnnie dla każdego z zapytań. Przebiega to w taki sam sposób jak w przypadku standardowego alogrytmu Apriori. Z każdym zapytaniem powiązane jest drzewo haszowe, w którym przechowywani są kandydaci. Warunek zatrzymania algorytmu jest taki jak w standardowym Apriori (brak możliwości wygenerowania kandyadtów w kolejnej iteracji), z tą różnicą, że musi być spełniony dla wszystkich zapytań ze zbioru. 
Zliczenie wystąpień kandyadatów jest realizowane dla wszystkich zapytań jednocześnie. Partycje bazy danych są odczytywane sekwencyjnie dla poszczególnych elementarnych predykatów selekcji danych. Powiększeniu ulegają liczniki kandydatów zawartych w analizowanej transakcji dla zapytań posiadającyh odwołania do danej partycji. Lista kandydatów zawiarających się w danej transakcji ustalana jest poprzez testowanie transakcji względem drzew haszowych. Należy tutaj zaznaczyć, że w przypadku gdzy kilka zapytań współdzieli dany elementarny predykat selekcji danych, to podczas zliczeń wystąpień kandydatów odczyt właściwej mu partycji jest wykonywany tylko raz. Zatem optymalizowane są odczyty współdzielonych przez zapytania fragmentów bazy danych, przy czym pozostałe kroki algorytmu Apriori pozostają niezmienione i są wykonywane oddzielnie dla każdego zapytania. 

\subsection{Common Candidate Tree \cite{WojciechowskiCCT}}
\label{c326}
Common Candidate Tree podobna do Common Counting. Również korzysta z oryginalnego Apriori i wykorzystuje strukturę drzewa haszowego. Różnica polega na tym, że zwiększony został stopień współbieżności przetwarzania. Uzyskano to dzięki współdzieleniu pamięciowej struktury drzewa składującego kandydatów. Jest to duża zaleta w porównaniu z Common Counting, gdyż - zamiast wielu - tworzone jest jedno drzewo haszowe o niezmiennej strukturze. Poza zachowaniem integracji odczytów współdzielonych możliwa jest integracja testowania czy w danej transakcji zawierają się kandydaci z poszczególnych zapytań. Realizacja tego alogrytmu wymagała rozszerzenia struktury kandydatów. W jej wyniku z każdym kandydatem związany został wektor liczników (jeden licznik dla jednego zapytania), a nie pojedyńczy licznik. Dodatkowo - dla rozróżnienia zapytań, które wygenerowały danego kandydata - dołączony został wektor flag logicznych przechowujący taką właśnie informację. Po wyłonieniu kadydatów są oni umieszczani w jednym zbiorze. Zbiór ten trafia do wspólnego drzewa haszowego. W tym kroku modyfikowane są również odpowiednie falgi. Samo generowanie kandydatów i selekcja zbiorów czestych nadal realizowane sa odrebnie dla poszczególnych zapytań. Zliczany jest natomiast zintegrowany zbiór kandydatów. Podczas tej fazy brani są pod uwagę tylko kandydaci wygenerowani przez zapytania odwołujące się do aktualnie odczytywanej partycji bazy danych i w przypadku gdy kandydat zawiera się w przetwarzanej transakcji, to zwiększa się liczniki kandydatów związne z tymi zapytaniami.
Eksperymenty \cite{WojciechwskiCCT} pokazały, że jest to alogrytm wydajniejszy i lepiej skalowany od Common Counting.
\chapter{Opracowane algorytmy}
\label{c4}

\section{Wstęp}
\label{c41}
Poniższy rozdział poświęcono przedstawieniu idei, opisu i cech zaprojektowanych algorytmów. Oba oparte są na opisanych w poprzednim rozdziale algorytmach przetwarzania zbiorów zapytań eksploracyjnych, tj. Common Counting (\cite{WojciechowskiCC}) i Common Candidate Tree (\cite{WojciechowskiCCT}). Implementacja Apriori wykorzystywana wewnątrz tych algorytmów jest zgodna z rozwiązaniem zaproponowanym przez Borgelta (\cite{Borgelt}).


\section{Przygotowanie danych}
\label{c42}
Zbiór danych, w którym poszukiwane są zbiory częste z użyciem Common Counting lub Common Candidate Tree musi spełniać kilka założeń. Krokiem wstępnym wykonania obu algorytmów jest zatem odpowiednie przygotowanie danych. Po pierwsze elementy każdej transakcji są sortowane leksykograficznie. Następnie niezbędne jest wyznaczenie zbioru elementarnych predykatów selekcji danych dla zbioru zapytań eksploracyjnych \(DMQ = \{dmq_1, dmq_2, \dots, dmq_n\}\). Przykładowo jeżeli relacja \(R\) posiada atrybut całkowitoliczbowy \(a\) oraz do wykonania są trzy zapytania eksploracyjne \(dmq_1=(R, 0 \leq a < 10, \emptyset, 4\%)\), \(dmq_2=(R, 5\leq a < 20, \emptyset, 2\%)\), \(dmq_3=(R, 0\leq a < 5 or 25\leq a < 30, \emptyset, 3\%)\), to w tym wypadku zbiór elementarnych predykatów selekcji danych będzie równy  \(S = \{0\leq a < 5, 5\leq a < 10, 10\leq a < 20, 25\leq a < 30\}\). Znając ten zbiór można zdefiniować \(DMQ\) zawierające rozłączne zapytania eksploracyjne. Po wykonaniu wszystkich zapytań z \(DMQ\) należy w odpowiedni sposób połączyć zebrane informacje i~zwrócić odpowiedzi na pierwotnie sformułowane zapytania. Tak przygotowany zbiór \(DMQ\) jest wejściem dla obu opracowanych algorytmów.

\subsection{Implementacja algorytmu Apriori}
\label{c43}

Implementacja Apriori zastosowanego wewnątrz CC i CCT jest inspirowana propozycja Borgelta \cite{Borgelt}. Zastosowane zostało drzewo prefiksowe jako struktura przechowująca kandydatów,dla których ustalane jest wsparcie. Drzewo generowane jest od korzenia. Na pierwszym poziomie znajdują się wszystkie możliwe zbiory jednoelementowe. Ich liczność odpowiada liczbie różnych elementów w zbiorze wszystkich elementów występujących w przetwarzanych transakcjach. Następnie wykonywane jest ustalanie wsparcia dla każdego wierzchołka. Kolejny poziom w drzewie generowany jest tylko z wierzchołków zawierających zbiory częste. Dzięki temu znacząco zmniejsza się rozmiar wygenerowanego drzewa. Dla pierwszego poziomu wsparcie jest po prostu sumą wystąpień poszczególnych elementów w transakcjach. Dla pozostałych poziomów Wsparcie wyznaczane jest metodą rekurencyjną liczenia rekurencyjnego (RC) (ang. \textit{recursive counting}). Metoda ta działa dla każdego wierzchołka w następujący sposób: (1) przejdź do dziecka wskazywanego przez krawędź z etykietą odpowiadającą pierwszemu elementowi transakcji i przetwarzaj dla niego pozostałe elementy transakcji w ten sam sposób oraz (2) pomiń pierwszy element transakcji i przetwarzaj dla danego wierzchołka pozostałe elementy. Gdy metoda znajdzie się na poziomie odpowiadającym aktualnie dodanym kandydatom, to zwiększany jest licznik wystąpień w danym wierzchołku i nie następuje dalsze przechodzenie w głąb drzewa. Procedura ta jest sekwencyjnie powtarzana dla każdej transakcji. Dodatkowo, w celu zmniejszenia liczby analizowanych transakcji, sprawdzane jest czy liczba elementów transakcji jest wystarczająca do osiągnięcia rozważanej głębokości drzewa - jeśli nie, to taka transakcja jest pomijana. Algorytm Apriori kończy się w momencie gdy nie udało się wygenerować kandydatów dla kolejnego poziomu drzewa.

\section{Common Counting z wykorzystaniem drzew prefiksowych (CCP)}
\label{c44}



\section{Common Candidate Tree z wykorzystaniem drzew prefiksowych (CCTP)}
\label{c45}

\chapter{Wyniki eksperymentów}
\label{c5}

\section{Wstęp}
\label{c51}
W poniższym rozdziale opisano przebieg przeprowadzonych eksperymentów oraz przedstawiono uzyskane wyniki wraz z ich interpretacją. Algorytmy testowane były na tych samych zbiorach danych. Tworzenie tych zbiorów odbywało się z wykorzystaniem generatora przyjmującego jako parametry liczbę transakcji (np. 10 tys., 100 tys.), średnią liczbę elementów w transakcji (np. 6), liczbę wzorców do odkrycia (np. 500), średni rozmiar wzorców częstych do odkrycia (np. 3), liczbę różnych elementów występujących w transakcjach (np. 1000, 10000) oraz nazwę pliku wyjściowego. Wygenerowany plik jest importowany do bazy danych, do której odwołuje się aplikacja podczas wykonywania algorytmów. Dane w bazie składowane są w jednej tabeli w postaci par \((id transakcji, id elementu)\). Dlatego też po odczycie tej tabeli składane są transakcje wykorzystywane w dalszym przetwarzaniu. Zbiór zapytań \(DMQ\) generowany był przed rozpoczęciem przetwarzania. Liczba transakcji obejmująca każde zapytanie \(dmq_i\) wynosi \(|T|/|DMQ|\), gdzie \(|T|\) - liczba wszystkich transakcji, \(|DMQ|\) - liczba wszystkich zapytań eksploracyjnych. 

\section{Opis infrastruktury}
\label{c52}
Algorytmy napisane zostały w języku Java, z wykorzystaniem narzędzia Maven oraz środowiska programistycznego Eclipse. Dane testowe generowane były za pomocą generatora GEN (\cite{AgrawalGEN}), a następnie wczytywane do bazy PostgreSQL, z której korzystała aplikacja. Testy przeprowadzone zostały na komputerze HP Envy 14 Notebook PC, z procesorem Intel Core i5-2410M 2x2.30GHz oraz 8GB pamięci RAM, pracującym pod kontrola systemu operacyjnego Microsoft Windows 7. 

\section{Wyniki}
\label{c53}
Poniżej zaprezentowano wyniki eksperymentów. Wszystkie uwzględnione czasy są średnią 4 dokonanych pomiarów dla dokładnie tych samych parametrów. Badania przeprowadzono dla dwóch zapytań eksploracyjnych z różnym poziomem nakładania się danych. Oba zapytania wykorzystywały ten sam próg wsparcia, mimo że nie jest to wymaganiem algorytmu. Jest to podyktowane chęcią możliwości lepszego zaobserwowania różnic w działaniu algorytmów niezależnie od wykorzystywanych progów dla poszczególnych zapytań. 

Zbiór danych 1.\newline
Parametry generatora dla drugiego zbioru danych przedstawia tabela \ref{table:firstDataSetParams}.
\begin{table}[h]
\begin{center}
	\begin{tabular}{| l | c |}
		\hline
		liczba transakcji & 100000 \\ \hline
		średnia liczba elementów w transakcji & 8 \\ \hline
		liczba różnych elementów & 1000 \\ \hline
		liczba wzorców (zbiorów częstych) & 1500 \\ \hline
		średnia liczba elementów we wzorcu & 4 \\ 
		\hline
	\end{tabular}
\end{center}
\caption{Parametry generatora dla pierwszego testowego zbioru transakcji.}
\label{table:firstDataSetParams}
\end{table}

\begin{figure}[h]
	\centering
	\includegraphics[width=0.8\linewidth]{figures/chart_100_2}
	\caption{Wykres dla pierwszego zbioru danych dla dwóch zapytań \(dmq\) z różnym stopnieniem nakładania się i \(minsup = 2\)}
	\label{fig:chart_100_2}
\end{figure}

\begin{figure}[h]
	\centering
	\includegraphics[width=0.8\linewidth]{figures/chart_100_1}
	\caption{Wykres dla pierwszego zbioru danych dla dwóch zapytań \(dmq\) z różnym stopnieniem nakładania się i \(minsup = 1\)}
	\label{fig:chart_100_1}
\end{figure}


Zbiór danych 2.\newline
Parametry generatora dla drugiego zbioru danych przedstawia tabela \ref{table:secondDataSetParams}.
\begin{table}[h]
	\begin{center}
		\begin{tabular}{| l | c |}
			\hline
			liczba transakcji & 50000 \\ \hline
			średnia liczba elementów w transakcji & 8 \\ \hline
			liczba różnych elementów & 1000 \\ \hline
			liczba wzorców (zbiorów częstych) & 1000 \\ \hline
			średnia liczba elementów we wzorcu & 4 \\ 
			\hline
		\end{tabular}
	\end{center}
	\caption{Parametry generatora dla drugiego testowego zbioru transakcji.}
	\label{table:secondDataSetParams}
\end{table}

\begin{figure}[h]
	\centering
	\includegraphics[width=0.8\linewidth]{figures/chart_50_2}
	\caption{Wykres dla drugiego zbioru danych dla dwóch zapytań \(dmq\) z różnym stopnieniem nakładania się i \(minsup = 2\)}
	\label{fig:chart_50_2}
\end{figure}

\begin{figure}[h]
	\centering
	\includegraphics[width=0.8\linewidth]{figures/chart_50_1}
	\caption{Wykres dla drugiego zbioru danych dla dwóch zapytań \(dmq\) z różnym stopnieniem nakładania się i \(minsup = 1\)}
	\label{fig:chart_50_1}
\end{figure}


\section{Dalsze prace}
\label{c54}
Adaptacje algorytmu Apriori oraz algorytmy przetwarzające zbiory zapytań eksploracyjnych nadal mogą być ulepszane i optymalizowane. Jednym z możliwych rozwiązań jest próba wykorzystania Common Counting w taki sposób, że na podstawie zbioru elementarnych predykatów selekcji danych dla zbioru zapytań eksploracyjnych \(S\) generowane są rozłączne zapytania \(dmq_i\), dla których algorytm wykonywany jest w ten sam sposób jaki opisano w \ref{c43}. Ostatnim krokiem byłaby wówczas integracja znalezionych zbiorów częstych w celu uzyskania odpowiedzi na oryginalne zapytania ze zbioru \(DMQ\). Takie podejście zmniejszyłoby rozmiary drzew przechowujących kandydatów podczas przetwarzania. Należałoby sprawdzić czy zysk ten jest większy niż koszt operacji generowania zbioru rozłącznych zapytań \(DMQ\). 

Innym zagadnieniem jest problem optymalizacji generowania kandydatów dla CCTP. Sortowanie wartości podczas dodawania jest czasochłonne, mimo że porównywane są jedynie dwie wartości zbioru elementów, a nie cała lista wartości. Mimo prób zastosowania innych struktur (m.in. zbioru czy też drzew haszowych) podczas tej operacji nie udało się uzyskać lepszego wyniku niż dla ostatecznie zastosowanego rozwiązania. 
\chapter{Wnioski i uwagi}

Whole conclusion for one page.

% All appendices and extra material, if you have any.
\cleardoublepage\appendix%
\chapter{Content of the DVD}

As an addition to this document, the DVD is attached. It provides some materials connected with the presented subject in electronic form for potential users or people, who would want to continue works on this topic. 

The DVD content consists of several items:

\begin{enumerate}
\item Item 1
\item Item 2
\item Item 3
\end{enumerate}



% Bibliography (books, articles) starts here.
\bibliographystyle{alpha}{\raggedright\sloppy\small\bibliography{bibliography}}

% Colophon is a place where you should let others know about copyrights etc.
\ppcolophon

\end{document}
